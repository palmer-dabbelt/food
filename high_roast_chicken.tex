\documentclass{recipe}

\begin{document}
\begin{recipe}{High Roast Chicken}
  \servings{4}

  \begin{ingredients}
    \ingredient{5}{lb}{roasting chicken}
    \ingredient{}{}{salt}
    \ingredient{}{}{sugar}
    \ingredient{3}{}{russet potatoes}
    \ingredient{}{}{vegetable oil}
    \ingredient{2}{tbsp}{butter}
    \ingredient{}{}{herbs de provence}
    \ingredient{}{}{olive oil}
  \end{ingredients}

  \begin{images}
    \begin{image}
      \includegraphics[width=\linewidth,trim=950px 800px 1050px 850px, clip=true]{high_roast_chicken-01.jpeg}
    \end{image}
  \end{images}

  \begin{steps}
  \item Brine the chicken for 90 minutes at room temperature.
  \item Thinly slice the potatoes, toss them in a bit of vegetable
    oil, and place them in the bottom of a roasting tray.
  \item Butterfly the chicken, spread some herbed butter under the
    skin, and place it on top of the potatoes (over a roasting tray).
  \item Rub some olive oil on the chicken skin for even browning.
  \item Roast at $500\degF$ for $40$ minutes, turning once for even
    browning.
  \end{steps}

  \begin{notes}
  \item This has been pretty much exactly copied from America's Test
    Kitchen.
  \item The potatoes were too oily, but I only had 2 in there so maybe
    that was the problem?  I'm thinking I don't actually need any fat
    on the potatoes to start, and I dripped a bunch while rubbing
    stuff on the chicken.
  \item The exposed potatoes burned, so I think I should have actually
    roasted at something more like $450\degF$, for about $45$ minutes
    (it was a bit over done).
  \item I think I probably should have put some salt on top of the
    chicken, the potatoes were a bit bland.
  \end{notes}
\end{recipe}
\end{document}
